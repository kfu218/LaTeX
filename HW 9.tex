\documentclass [12pt]{article}
\usepackage [utf8]{inputenc}
\usepackage [russian]{babel}
\usepackage{mathtools}
\begin{document}

	\begin{center}
		\textbf{C общей нумерацией}
	\end{center}

	\begin{equation}\label{eq:syst}
		\left\{
			\begin{array}{rcl}
			54x^2+32y^2&=&73,\\
			x+y&=&3.\\
		\end{array}
		\right.
	\end{equation}

	\begin {center}
		\textbf{Формула лесенкой}
	\end {center}

	\begin{multline}
		p_n(x) =(x - x_0)(x - x_1)
		\cdots \\
		(x - x_{j-1})(x - x_{j+1})
		\cdots \\
		(x - x_{n-1})(x - x_n)
	\end{multline}

	\begin {center}
  		\textbf{Формулы в тексте}
	\end {center}
	а теперь попробуем формулы, к примеру вот внутритекстовое квадратное
уравнение $ax^2+bx+c=0$ \ формула кончилась

	Производная от
	\begin {math}B_2’ = B_2^\prime \end{math}

	\begin {center}
  		\textbf{Формулы сокращенного умножения}
	\end {center}

	$$
		(a+b)^2=a^2+2ab+b^2 \ \text{квадрат суммы} 
	$$

	$$
		(ab)^2=a^2-2ab+b^2 \ \text{квадрат разности}
	$$
	
	$$ 
  		a^2-b^2=(a-b)^2(a+b)^2\ \text{разность квадратов}
	$$
\end{document}