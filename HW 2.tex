\documentclass[11pt]{report}
\usepackage[utf8]{inputenc}
\usepackage[russian, english]{babel}
\usepackage{ulem}\normalem
\begin{document}


	{\small {\bf Rails is a web application development framework written in the Ruby language. It is designed to make programming web applications easier by making assumptions about what every developer needs to get started. It allows you to write less code while accomplishing more than many other languages and frameworks. Experienced Rails developers also report that it makes web application development more fun.

Rails is opinionated software. It makes the assumption that there is the "best" way to do things, and it's designed to encourage that way - and in some cases to discourage alternatives. If you learn "The Rails Way" you'll probably discover a tremendous increase in productivity. If you persist in bringing old habits from other languages to your Rails development, and trying to use patterns you learned elsewhere, you may have a less happy experience.}}

	\uline{Compiling CoffeeScript and JavaScript asset compression requires you have a JavaScript runtime available on your system, in the absence of a runtime you will see an execjs error during asset compilation. Usually Mac OS X and Windows come with a JavaScript runtime installed. Rails adds the therubyracer gem to the generated Gemfile in a commented line for new apps and you can uncomment if you need it. therubyrhino is the recommended runtime for JRuby users and is added by default to the Gemfile in apps generated under JRuby. You can investigate all the supported runtimes at ExecJS.}

	{\LARGE {\it To stop the web server, hit Ctrl+C in the terminal window where it's running. To verify the server has stopped you should see your command prompt cursor again. For most UNIX-like systems including Mac OS X this will be a dollar sign \$. In development mode, Rails does not generally require you to restart the server; changes you make in files will be automatically picked up by the server.}}
\end{document}