\documentclass{article}
\usepackage[utf8]{inputenc}
\usepackage[russian]{babel}
\usepackage{amsthm}
\newenvironment{myquotation}% Обратите внимание: "%" маскирует новую строку
	{\begin{center}\begin{itshape}}%
	{\end{itshape}\end{center}}
\newtheorem{theorem}{Theorem}[section]
\begin{document}
	\begin{myquotation}
		{\bf Эрл Мартов}\\
		"Ромб".\\
		Мы -\\
		Среди тьмы.\\
		Глаз отдыхает.\\
		Сумрак ночи живой.\\
		Сердце жадно вздыхает.\\
		Шепот звезд долетает порой,\\
		И лазурные чувства теснятся толпой.\\
		Все забылося в блеске росистом.\\
		Поцелуем  душистым\\
		Поскорее блесни!\\
		Снова шепни,\\
		Как тогда:\\
		"Да!"
	\end{myquotation}
	
	\newpage
		
	\begin{theorem}[Pythagorean theorem]
		\label{pythagorean}
		This is a theorema about right triangles and can be summarised in the next 
equation 
		\[ x^2 + y^2 = z^2 \]
	\end{theorem}
	
	\begin{proof}
		Here will proof.
	\end{proof}
\end{document}